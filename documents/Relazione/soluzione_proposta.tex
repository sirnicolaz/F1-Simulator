	%%%Architettura alto livello%%%
\section{Architettura ad alto livello}
Nella seguente sezione verr\`{a} illustrata l'architettura ad alto livello del sistema sviluppato, 
escludendo i dettagli implementativi e legati al linguaggio. Viene fornito un diagramma delle componenti per mostrare
molto ad alto livello l'architettura del sistema.\\
\begin{center}
\begin{figure}[h!]
	\includegraphics[scale=0.50]{img/ComponentDiagram.jpg}
	\caption{Diagramma delle componenti}
\end{figure}
\end{center}
% Componenti del sistema
\subsection{Componenti di sistema}
Le principali componenti del sistema sono:
\subsubsection{Competition}
La \emph{Competition} \`{e} l'unit\`{a} atta ad orchestrare l'avvio e la conclusione della corsa. Tale componente, dunque, \`{e} stata concepita per 
offrire le seguenti funzionalit\`{a}:
\begin{itemize}
\item Configurazione parametri di gara:
	\begin{itemize}
		\item numero di giri;
		\item numero di concorrenti;
		\item circuito;
	\end{itemize}
\item Gestione della sessione di iscrizione e accettazione concorrenti (configurati a livello della componente \emph{Box})
\item Avvio delle componenti necessarie al monitoraggio della gara (quali ad esempio \emph{Monitor})
\item Avvio controllato della competizione vera e propria nel momento in cui tutti i prerequisiti di inizio sono soddisfatti, ovvero:
\begin{itemize}
\item la competizione \`{e} stata configurata correttamente;
\item le componenti di controllo e gestione della competizione sono attive e in attesa di comandi;
\item i concorrenti sono stati correttamente registrati alla competizione e in attesa di partire;
\end{itemize}
\end{itemize}
\subsubsection{Competitor}
Il \emph{Competitor} \`{e} l'entit\`{a} pensata ad svolgere la gara. \`{E} caratterizzato dalle seguenti sotto-componenti:
\begin{itemize}
\item \textbf{Auto}, ovvero tutte le caratteristiche fisiche legate all'auto, ovvero:
	\begin{itemize}
		\item motore;
		\item capacit\`{a} del serbatoio;
		\item massima accelerazione;
		\item massima velocit\`{a};
		\item gomme montate (mescola, modello, tipo);
		\item livello usura gomme;
		\item livello della benzina nel serbatoio;
	\end{itemize}
\item \textbf{Guidatore}, cio\`{e} le informazioni che descrivono pi\`{u} dettagliatamente il concorrente in gara:
	\begin{itemize}
		\item nome e cognome pilota;
		\item nome scuderia
	\end{itemize}
\item \textbf{Strategia}, ovvero la strategia che sta adottando il pilota, suggerita dai box e dinamica nel corso della gara:
	\begin{itemize}
		\item stile di guida, variabile tra conservativo, normale, aggressivo, a seconda dello stato della macchina e delle
			previsioni fatte dai box
		\item numero di lap prima del pit stop
		\item addizionalmente, quando viene fatto un pit stop, la strategia determina anche quali siano le nuove gomme da montare,
		 	la quantit\`{a} di benzina da avere nel serbatoio e il tempo impiegato per il pit stop.
	\end{itemize}
\end{itemize}
Tutte queste informazioni insieme creano quello che viene definito il concorrente di gara. 
Tali informazioni verranno poi usate nel corso della gara per:
	\begin{itemize}
		\item scegliere al momento giusto la miglior traiettoria da seguire, in base alla presenza o meno di altri concorrenti
			nelle vicinanze e alla difficolt\`{a} del tratto;
		\item fornire costantemente aggiornamenti sul suo stato (tramite una parte del modulo \emph{Stats}, informando il computer di bordo
			riguardo a:
			\begin{itemize}
				\item livello di usura gomme;
				\item livello di benzina;
				\item checkpoint attraversato con tempo di arrivo;
				\item insieme al checkpoint verranno aggiornate le informazioni relative a settore e lap;
				\item velocit\`{a} massima raggiunta.
			\end{itemize}
		\item contattare ad ogni giro il box per ottenere una strategia aggiornata;
		\item se suggerito dai box, effettuare un pitstop;
		\item ritirarsi dalla gara una volta che le condizione dell'auto non permettano di poter correre ulteriormente;
		\item banalmente, continuare a correre fino alla fine delle lap prestabilite, dopodich\`{e} fermarsi.
	\end{itemize}
\subsubsection{Circuit}
Il circuito \`{e} una risorsa finalizzata ad offrire il piano su cui svolgere la competizione. \`{E} condivisa fra tutti i concorrenti in gara e
offre un insieme di funzionalit\`{a} per poter conoscere le caratteristiche dei vari tratti della pista (compresi i concorrenti presenti
al momento dell'attraversamento). \`{E} composto dalle seguenti sottocomponenti:
\begin{itemize}
\item \textbf{Checkpoint}:
i \emph{Checkpoint} rappresentano punti di arrivo intermedi del circuito. Come una suddivisione in fette da 1 secondo possono discretizzare 1 minuto,
cos\`{i} i \emph{Checkpoint} discretizzano il circuito. Ogni \emph{Checkpoint} introduce un tratto della pista potenzialmente diverso da quello precedente.
Per esempio, il $\emph{Checkpoint}_n$ potrebbe essere il punto di entrata di un tratto della pista accessibile ad un numero massimo di 4 concorrenti insieme,
mentre il successivo $\emph{Checkpoint}_{n+1}$ potrebbe esporre un tratto pi`{u} stretto e quindi accessibile solo a 2 concorrenti. Schematizzando, 
il \emph{Checkpoint} \`{e} caratterizzato da:
\begin{itemize}
\item \textbf{molteplicit\`{a}}: ovvero il numero di concorrenti che possono trovarsi contemporaneamente nel tratto a seguire;
\item \textbf{posizione nella pista}: un \emph{Checkpoint} pu\`{u} essere il traguardo, l'inizio del settore, la fine di un settore, all'uscita dei box, l'entrata
					per i box, i box oppure un punto intermedio fra altri due \emph{Checkpoint};
\item \textbf{tempi di arrivo}: ogni \emph{Checkpoint} tiene traccia dell'istante in cui un concorrente ci \`{e} passato sopra.
\end{itemize}
\item \textbf{Path}: rappresenta una delle possibili traiettorie da usare per andare da un \emph{Checkpoint} a quello successivo. La traiettoria presenta
	un numero di \emph{Path} uguale alla molteplicit\`{a} del \emph{Checkpoint} che la precede. Ogni \emph{Path} \`{e} descritto da:
	\begin{itemize}
		\item lunghezza
		\item angolo
		\item grip, ovvero l'aderenza sul tratto
	\end{itemize}
Normalmente i path appartenenti allo stesso tratto differiscono di poco.
\item \textbf{Iteratore}: l'unit\`{a} permette di sapere la struttura della pista. Si suppone venga usata per sapere quale \emph{Checkpoint} ne segue un altro,
	oppure per sapere dove sia il \emph{Checkpoint} di inizio box. 
\end{itemize}
\subsubsection{Stats}
Questa componente mantiene la storia della gara e offre un insieme di funzionalit\`{a} che permettono di elaborare tali dati per offrirne differenti viste:
	\begin{itemize}
		\item migliori performance in un determinato istante di tempo
		\item classifica aggiornata per istante di tempo
		\item informazioni sui concorrenti relative ad un particolare lap, checkpoint o settore (in una specifica lap);
	\end{itemize}
\subsubsection{Box}
Il \emph{Box} \`{e} l'entit\`{a} che si occupa di gestire la configurazione e la corsa di un concorrente. Durante la competizione, il \emph{Box} verifica costantemente
lo stato dell'auto e fornisce eventuali cambi di strategia se ritenuto opportuno. Inoltre decide quando i pitstop del concorrente con tutte le caratteristiche 
ad esso legate, quali:
	\begin{itemize}
		\item benzina da aggiungere nel serbatoio
		\item gomme da montare
	\end{itemize}
Ogni \emph{Box} \`{e} caratterizzato da uno fra 4 tipi di strategia, diversi per grado di ``ottimismo'' nelle valutazioni e nei calcoli dati lo stato della 
macchina, le medie calcolate e lo stile di guida del concorrente:
	\begin{enumerate}
		\item \textbf{Cautious}: cauto, sottostima il numero di giri ancora fattibili;
		\item \textbf{Normal}: stima abbastanza realistica delle possibilit\`{a} del concorrente, considera anche un margine di errore nei calcoli
			per effettuare una valutazione;
		\item \textbf{Risky}: le stime vengono effettuate in base a calcoli esatti che di solito non tengono in considerazione fattori che nella
			realt\`{a} possono incidere in modo negativo;
		\item \textbf{Fool}: nella realt\`{a} normalmente non si arriva a tanto, ma per fini di test \`{e} stato inserito anche un tipo di strategia
			che sovrastima le possibilit\`{a} del concorrente, portandolo a squalifica quasi certa.
	\end{enumerate}
Ci\`{o} che il box suggerisce al concorrente durante la gara \`{e}:
	\begin{itemize}
		\item stile di guida. Verr\`{a} suggerito uno stile pi\`{u} conservativo se i consumi si sono rivelati maggiori del previsto e viceversa;
		\item numero di lap al pitstop
	\end{itemize}
Il \emph{Box} riceve informazioni sullo stato del concorrente alla fine di ogni settore e ricalcola la strategia alla fine del secondo settore. Il concorrente
richiede la nuova strategia al box in prossimit\`{a} del checkpoint dove \`{e} possibile proguire o andare ai box.\\
\`{E} sembrato pi\`{u} realistica la scelta di non calcolare la strategia alla fine del terzo settore, perch\`{e} si suppone che nella realt\`{a} non si possa essere
cos\`{i} veloci da calcolare una nuova strategia istantaneamente alla fine del circuito con i dati del terzo settore. \`{E} piuttosto pi\`{u} probabile che 
qualunque cambio di strategia o richiesta di rientro ai box venga stabilita gi\`{a} alla fine del secondo settore. In modo che in prossimit\`{a} dei box il concorrente
possa ottenere l'informazione istantaneamente e possa quindi decidere come e dove procedere.
\subsubsection{Monitor}
La componente \emph{Monitor} e costituita dall'insieme delle unit\`{a} concepite per esporre le informazioni e i dati prodotti a basso livello
dal simulatore. Ne esistono due tipi:
\begin{itemize}
	\item \textbf{monitor di competizione}: utilizzato per offrire informazioni riguardanti la gara e i singoli concorrenti;
	\item \textbf{monitor di box}: utilizzato per esporre i dati relativi alle computazioni del box, e le informazioni grezze legate al
		concorrente appartenente alla sua scuderia e i dati rielaborati di settore in settore per formulare nuove strategie.
\end{itemize}
\subsubsection{Screen} 
Gli \emph{Screen} sono la parte pi\`{u} alta del sistema. Servono a mettere in connessione l'utente finale e la parte logica del simulatore.
Tramite gli \emph{Screen} \`{e} possibile ricevere aggiornamenti grafici (di base) sullo stato di avanzamento della simulazione sotto il punto di vista 
dei singoli box o della gara complessiva.
\subsubsection{Configurator}
Questa componente offre la possibilit\`{a} di configurare le parti parametriche del sistema, ovvero:
\begin{itemize}
	\item concorrenti 
	\item competizione (i parametri elencati nella sezione \emph{Competition}
	\item box
\end{itemize}
Il \emph{Configurator} ha inoltre l'onere di inviare i parametri cos\`{i} configurati alle componenti opportune.

%\subsubsection{Radio}
%La componente radio \`{e} stata creata per gestire la comunicazione remota fra le componenti distribuite.\\
%Il sistema inizialmente \`{e} stato implementato con supporto solo all'esecuzione locale. Si conoscevano le componenti che sarebbero state distribuite,
%ma il processo di separazione \`{e} avvenuto successivamente. Quindi per agevolare tale processo si \`{e} pensato di demandare l'onere della comunicazione 
%tra queste componenti a unit\`{a} specializzate, chiamate appunto radio. In questo modo \`{e} stato pi\`{u} soft il passaggio da locale a distribuito,
%che ha richiesto di rivedere solo le componenti \emph{Radio}.\\\\
%\emph{\textsc{Nota Bene}}:\\
%La componente \emph{Radio} costituisce un layer di supporto alla comunicazione 
%distribuita delle componenti. Considerando che a questo livello del documento non \`{e} ancora il momento di scendere nei dettagli fino a discutere 
%della distribuzione, si pensa sia sufficiente averla nominata nella descrizione delle componenti. Verr\`{a} ripresa
%in considerazione successivamente nel documento. Inserendola ora nei diagrammi delle componenti rischierebbe di confondere la visione di insieme
%del sistema.
% Interazione fra le componenti
\subsection{Interazione fra le componenti}
Nella seguente sezione verranno spiegate le interazione principali fra le componenti.\\\\
\subsubsection{Configurator-Competition}
\begin{center}
\begin{figure}[h!]
	\includegraphics[scale=0.55]{img/InteractionDiagram/Implementation_Diagram__ConfiguratorCompetition.jpg}
\caption{Interface component diagram - Configurator/Competition}
\end{figure}
\end{center}
Al livello pi\`{u} alto della fase di configurazione c'\`{e} la componente \emph{Configurator}, la quale viene utilizzata per impostare i parametri
relativi alla \emph{Competition}. Tali parametri vengono prima impostati dall'utente (o letti da file) tramite le interfacce \textbf{SetupCompetitor},
\textbf{SetupCompetition} e \textbf{SetupCircuit} e successivamente inviati alla componente \emph{Competition} per l'inizializzazione.\\
\textbf{RegisterCompetitor} \`{e} utilizzata una volta per ogni competitor da iscrivere.
\subsubsection{Competition-Competitor}
\begin{center}
\begin{figure}[h!]
	\includegraphics[scale=0.55]{img/InteractionDiagram/Implementation_Diagram__CompetitionCompetitor.jpg}
\caption{Interface component diagram - Competition/Competitor}
\end{figure}
\end{center}
L'interazione fra la \emph{Competition} e il \emph{Competitor} avviene, come per le altre interazioni \emph{Competition}-componente in fase di configurazione
e avvio. La \emph{Competition} si occupa di ricevere i parametri relativi a ogni \emph{Competitor} dal \emph{Configurator} (come verr\`{a} spiegato fra poco) per 
poi inizalizzare il competitor. Quando tutti i concorrenti sono pronti e anche i \emph{Box}, l'interfaccia \textbf{Start} verr\`{a} utilizzata per dare il
via ai concorrenti.
\subsubsection{Competition-Monitor}
\begin{center}
\begin{figure}[h!]
	\includegraphics[scale=0.55]{img/InteractionDiagram/Implementation_Diagram__CompetitionMonitor.jpg}
\caption{Interface component diagram - Competition/Monitor}
\end{figure}
\end{center}
L'interazione \emph{Competition}-\emph{Monitor} riguarda solo una piccola fase dell'inizializzazione, processo che verr\`{a} esplicato in dettaglio in seguito.
A concorrenti registrati, la \emph{Competition} sfruttera un interfaccia offerta dal monitor per sapere quando tutti i \emph{Box} sono pronti e avviati.
L'interfaccia \`{e} l'unica illustrata in figura.
\subsubsection{Competition-Circuit}
\begin{center}
\begin{figure}[h!]
	\includegraphics[scale=0.55]{img/InteractionDiagram/Implementation_Diagram__CompetitionCircuit.jpg}
\caption{Interface component diagram - Competition/Circuit}
\end{figure}
\end{center}
Anche in questo caso, l'interazione \emph{Competition}-\emph{Circuit} si ha in fase di inizializzazione quando le due componenti entrano in contatto
per la configurazione. La \emph{Competition} inizializza cio\`{e} il circuito impostando i parametri che lo caratterizzano, quali:
	\begin{itemize}
		\item numero di checkpoint
		\item caratteristiche dei tratti fra checkpoint (lunghezza, angolo ...)
		\item posizione dei checkpoint di entrata e uscita box
		\item posizione del checkpoint traguardo
	\end{itemize}
\subsubsection{Competition-Stats}
\begin{center}
\begin{figure}[h!]
	\includegraphics[scale=0.55]{img/InteractionDiagram/Implementation_Diagram__CompetitionStats.jpg}
\caption{Interface component diagram - Competition/Stats}
\end{figure}
\end{center}
La componente \emph{Stats} ottiene i parametri di configurazione in fase di inizalizzazione dalla \emph{Competition}, tramite
\textbf{InitilizeStatistics}.
\subsubsection{Competitor-Stats}
\begin{center}
\begin{figure}[h!]
	\includegraphics[scale=0.55]{img/InteractionDiagram/Implementation_Diagram__CompetitorStats.jpg}
\caption{Interface component diagram - Competitor/Stats}
\end{figure}
\end{center}
La componente \emph{Stats} offre al \emph{Competitor} l'interfaccia \textbf{AddInfo} che il concorrente utilizza per fornire costantemente a \emph{Stats}
dati aggiornati riguardo alla gara in corso (dati relativi al singolo concorrente). Ad ogni checkpoint quindi il concorrente manda un aggiornamento
a \emph{Stats} che poi verranno utilizzate per effettuare calcoli di insieme riguardo alla gara o per essere mandate a chi le richiedesse.
\subsubsection{Competitor-Circuit}
\begin{center}
\begin{figure}[h!]
	\includegraphics[scale=0.55]{img/InteractionDiagram/Implementation_Diagram__CompetitorCircuit.jpg}
\caption{Interface component diagram - Competitor/Circuit}
\end{figure}
\end{center}
L'interazione \emph{Competitor}-\emph{Circuit} avviene durante lo svolgimento della competizione. Il concorrente sfrutta le interfacce del \emph{Circuit}
per ottenere informazioni sul circuito e ``informarlo'' degli spostamenti nel corso della gara.\\
\textbf{GetCheckpoint} garantisce che il concorrente ottenga sempre il checkpoint corretto a seconda della posizione corrente.
\textbf{GetPathCondition} permette di ottenere informazioni sulle caratteristiche statiche e dinamiche della tratto da attraversare. Le caratteristiche 
dinamiche sono legate ai concorrenti attualmente presenti sul tratto.
\textbf{CrossSegment} assicura che il tratto possa essere attraversato senza collisioni e che il \emph{Circuit} possa tracciare l'avvenuto passaggio 
dell'auto.
\subsubsection{Monitor-Stats}
\begin{center}
\begin{figure}[h!]
	\includegraphics[scale=0.55]{img/InteractionDiagram/Implementation_Diagram__MonitorStats.jpg}
\caption{Interface component diagram - Monitor/Stats}
\end{figure}
\end{center}
La componente \emph{Monitor} si appoggia a \emph{Stats} per poter reperire le informazioni da essere esposte. Per questo motivo
\emph{Stats} offre un insieme di interfacce finalizzate a fornire i dati grezzi di competizione sotto forma di viste utili alla ``pubblicazione''.\\
\textbf{GetBestPerformance} fornisce i migliori tempi relativi a settori e giro.\\
\textbf{GetCompetitorInfo} reperisce informazioni legate al singolo concorrente, come ad esempio lo stato della macchina ad un determinato istante.\\
\textbf{GetClassification}, come dice il nome, ritorna informazioni legate alla classifica.\\
\textbf{GetCompetitionStatus} espone informazioni legate alla competizione nel suo insieme, come ad esempio la posizione dei concorrenti nel circuio
in un determinato istante di tempo.
\subsubsection{Configurator-Box}
\begin{center}
\begin{figure}[h!]
	\includegraphics[scale=0.55]{img/InteractionDiagram/Implementation_Diagram__ConfiguratorBox.jpg}
\caption{Interface component diagram - Configurator/Box}
\end{figure}
\end{center}
La componente \emph{Configurator} offre anche la possibilit\`{a} di configurare un \emph{Box}. Per questo motivo il \emph{Box} espone un'interfaccia
da utilizzare per sottomettere i parametri di configurazione. 
\subsubsection{Box-Monitor}
\begin{center}
\begin{figure}[h!]
	\includegraphics[scale=0.55]{img/InteractionDiagram/Implementation_Diagram__BoxMonitor.jpg}
\caption{Interface component diagram - Box/Monitor}
\end{figure}
\end{center}
La prima interazione che la componente \emph{Box} ha con il \emph{Monitor} \`{e} in fase di inizializzazione della gara. 
La sequenza di azioni verr\`{a} esplicata
dettagliatamente in seguito, per ora basti sapere che il \emph{Box}, una volta configurato e pronto per monitorare 
la gara ed elaborare i dati del suo
concorrente, dovr\`{a} mandare una notifica tramite la componente \emph{Monitor} utilizzando \textbf{NotifyBoxReadyState}.\\
Le rimanenti due interfacce offerte dal \emph{Monitor} al \emph{Box} sono utilizzate per reperire informazioni sullo stato del 
concorrente (livello di benzina rimasta, usura gomme...) e
aggiornamenti riguardo al posizionamento e tempi del concorrente durante la gara.\\
Vi sono poi altre due interfacce offerte dal \emph{Box} al \emph{Monitor}. Per quanto possa sembrare un po' paradossale, l'architettura
acquisisce senso se si pensa che la componente \emph{Monitor}, ad alto livello, \`{e} stata pensata per pubblicare informazioni. Tali
informazioni possono riguardare il singolo concorrente e quindi essere utili al \emph{Box}. Altre possono invece riguardare il \emph{Box} e 
i suoi calcoli per essere esposte ad un utente (per esempio). Le due interfacce offerte dal \emph{Box} infatti servono per ottenere
aggiornamenti sulle operazioni interne del \emph{Box} qualora essere dovessero essere esposte ad un qualunque client. Come vedremo pi\`{u}
in dettaglio in seguito, questa componente \`{e} in realt\`{a} costituita da due sottocomponenti, una dedicata a \emph{Competition} e l'altra a
\emph{Box}.
\subsubsection{Screen-Monitor}
\begin{center}
\begin{figure}[h!]
	\includegraphics[scale=0.55]{img/InteractionDiagram/Implementation_Diagram__ScreenMonitor.jpg}
\caption{Interface component diagram - Screen/Monitor}
\end{figure}
\end{center}
La componente \emph{Screen} comunica con il \emph{Monitor} per ottenere informazioni utili da esporre graficamente all'utente. Tali
informazioni possono riguardare la competizione in senso globale, oppure essere legate ai singoli concorrenti e box.
\subsubsection{Competitor-Box}
\begin{center}
\begin{figure}[h!]
	\includegraphics[scale=0.55]{img/InteractionDiagram/Implementation_Diagram__CompetitoBox.jpg}
\caption{Interface component diagram - Competitor/Box}
\end{figure}
\end{center}
Man mano che la competizione procede, il \emph{Box} colleziona i dati di gara del rispettivo concorrente per calcolarne medie e statistiche. 
A partire da queste informazioni produce una nuova strategia ogni giro. Per questo offre un'interfaccia \textbf{RequestStrategy} che il concorrente
utilizza nel corso della gara per ottenere suggerimenti utili per proseguire. La strategia fornita dal \emph{Box} potrebbe anche richiedere 
un pitstop per un rifornimento benzina e cambio gomme.
% Strategia adottata per la correttezza temporale
\subsubsection{Strategia di simulazione}
\label{strategia_simulazione}
In questa sezione viene spiegata la strategia che \`{e} stata adottata per ottenere una simulazione realistica della gara. Le problematiche da 
affrontare sono gi\`{a} state discusse nel capitolo \ref{problematiche_concorrenza}.\\ 
La soluzione prevede l'esistenza di concorrenti che simultaneamente percorrono il circuito e un insieme di entit\`{a} di supporto per evitare 
i problemi visti. Tali entit\`{a} sono:
\begin{itemize}
  \item \textbf{coda al checkpoint}: ogni checkpoint introduce tratti di circuito con caratteristiche diverse rispetto a quello precedente.
	      \`{E} quindi possibile che 
  \item \textbf{istante di arrivo}:
  \item \textbf{istante limite di arrivo previsto}:
  \item \textbf{istante di liberazione traiettoria}:
\end{itemize}
% Dimostrazione dell'assenza di stallo
\subsubsection{Assenza di stallo}
\newpage
%%% Architettura in dettaglio %%
\section{Architettura in dettaglio}
Si spiegher\`{a} ora con maggior dettaglio l'architettura di sistema, esplicando come le principali classi implementate svolgono le funzionalit\`{a}
esposte dalle interfacce illustrate nel capitolo precedente.
\subsection{Diagrammi delle classi}
\subsubsection{Competition}
\begin{center}
\begin{figure}[h!]
	\includegraphics[scale=0.50]{img/ClassDiagrams/CompetitionClassDiagram.jpg}
\caption{Class diagram - Competition}
\end{figure}
\end{center}
La \emph{Competition} \`{e}, come gi\`{a} accennato, una componente di init.\\
Il tutto ha inizio a partire da \textbf{Main\_Competition} che, come dice il nome, \`{e} l'unit\`{a} di avvio.\\
\textbf{Main\_Competition} si occupa di istanziare gli oggetti necessari all'avvio e configurazione della competizione. Per la configurazione
vengono istanziati \textbf{RegistrationHandler} e \textbf{CompetitionConfigurator} (della componente \emph{Configurator}).
Per l'avvio invece \textbf{Starter}. \\
Ad orchestrare la scena, un unica istanza del \textbf{Synch\_Competition} condivisa fra le altre 3 unit\`{a}. Il loro scopo \`{e}:
\begin{itemize}
\item \textbf{Synch\_Competition}, \`{e} una risorsa protetta che gestisce l'accesso in mutua esclusione alla configurazione della competizione
e dell'inizializzazione dei concorrenti. La risorsa assicura tramite \emph{entry} a guardia booleana
che avvengano in ordine prima la configurazione
della competizione (da parte del \textbf{CompetitionConfigurator}) e poi la registrazione dei concorrenti (per opera del \textbf{RegistrationHandler}).
Questo vincolo \`{e} dato dal fatto che alcune impostazioni di competizioni devono essere fornite ai box dopo la configurazione del concorrente, 
come ad esempio il numero di lap. Inoltre fra i parametri configurabili vi \`{e} anche il numero massimo di partecipanti alla gara. Di conseguenza
\`{e} prima necessario conoscere il limite per poi poter regolare il flusso di registrazioni.\\
Durante la fase di configurazione (metodo \underline{Configure}), vengono inizializzati \emph{Circuit} e \emph{Stats} tramite le interfacce
illustrate nel diagramma.\\
A configurazione di competizione avvenuta, verr\`{a} aperta le entry \underline{Register\_NewCompetitor}, utilizzata dal \textbf{RegistrationHandler}
per registrare i vari concorrenti. Ad ogni invocazione verr\`{a} inizializzato un concorrente (\textbf{TaskCompetitor}) con un iteratore al circuito
e le impostazioni passate in input. Il task del concorrente cos\`{i} istanziato rimarr\`{a} in attesa dello ``start''.\\
Oltre alla funzionalit\`{a} di configurazione, questa classe offre anche la funzionalit\`{a} di avvio (metodo \underline{Start}).
Tale entry si aprir\`{a} solo quando tutti i concorrenti previsti sono stati iscritti. Viene invocata dall'unit\`{a} \textbf{Starter}.
Il metodo mette il task richiedente in attesa sulla risorsa \textbf{StartHandler} (componente \emph{Monitor}) sull'entry \underline{MonitorBoxReadyState}.
Quanto tutti i box avranno dato il loro ok (maggiori dettagli a seguire), il thread potr\`{a} continuare la sua esecuzione e passare allo \underline{Start}
di tutti i concorrenti in attesa di partire.
\item \textbf{Starter} \`{e} il task finalizzato a gestire l'avvio della competizione. Una volta avviato dal main, il task utilizza il metodo \underline{Ready}
offerto dal package \emph{Competition} per manovrare l'avvio. All'interno del metodo, prima si mette in attesa che tutti i concorrenti
si siano iscritti (utilizzando il metodo \underline{Wait} del singleton di \textbf{Synch\_Competition}) per poi invocare il metodo \underline{Start} dello
stesso \textbf{Synch\_Competition}, descritto poche righe pi\`{u} sopra.
\item \textbf{RegistrationHandler} \`{e} l'oggetto che rimane in attesa di concorrenti. Pi\`{u} precisamente, \`{e} un server dedicato ad accogliere
le richieste di registrazione dei concorrenti. Con il supporto di Polyorb \`{e} possibile invocare questo oggetto da remoto. Ad ogni richiesta,
vengono salvati i parametri di configurazione in un file xml il cui nome e locazione verranno passati a \textbf{Synch\_Competition} per
i effettuare il resto delle procedure di inizializzazione del concorrente. Di ritorno ci saranno l'ID del concorrente, il numero di lap,
la lunghezza del circuito e il corbaloc 
del \textbf{Competition\_Monitor} che il \emph{Box} dovr\`{a} utilizzare per rimanere sincronizzato sugli sviluppi del rispettivo concorrente.
\end{itemize}
\subsubsection{Competitor}
\begin{center}
\begin{figure}[h!]
	\includegraphics[scale=0.50]{img/ClassDiagrams/CompetitorClassDiagram.jpg}
\caption{Class diagram - Competitor}
\end{figure}
\end{center}

\newpage
\subsubsection{Circuit}
\begin{center}
\begin{figure}[h!]
	\includegraphics[scale=0.50]{img/ClassDiagrams/CircuitClassDiagram.jpg}
\caption{Class diagram - Circuit}
\end{figure}
\end{center}
Il circuito prende forma a partire da \textbf{Racetrack}. Questa unit\`{a} si occupa di leggere e parsare il file xml dato in input e ricavarne
i parametri per la creazione del circuito. Il file XML descrittore del circuito elenca i checkpoint presenti in ogni settore (che sono
3 costanti) e per ognuno specifica le caratteristiche del tratto di pista a seguire:
\begin{itemize}
  \item lunghezza
  \item angolo
  \item molteplicit\`{a} (numero di concorrenti che possono attraversare contemporaneamente il tratto)
  \item grip, ovvero aderenza sul tratto
\end{itemize}
Vi sono inoltre 3 checkpoint che dovranno avere uno dei seguenti attributi booleani:
\begin{itemize}
\item goal, ovvero il checkpoint \`{e} il primo della pista
\item prebox, ovvero dal checkpoint \`{e} possibile raggiungere i box
\item exitbox, ovvero il checkpoint di arrivo una volta vuori dalla corsia dei box
\end{itemize}
Dati questi parametri, vengono inizalizzati tutti i checkpoint stabiliti ed inseriti in un array chiamato \textbf{Racetrack}. Questo
sar\`{a} l'array iterato dal \textbf{Race\_Iterator}.\\

Ogni checkpoint \`{e} del tipo \textbf{Checkpoint}, il quale contiene tutte le informazioni elencate in precedenza oltre ad una coda per gestire
i concorrenti che tentano di accedere al tratto. Il \textbf{Checkpoint} \`{e} poi inserito in una struttura protetta denominata 
\textbf{Checkpoint\_Synch} finalizzata a regolare l'accesso in mutua esclusione. La risorsa inoltre incapsula la lista di \textbf{Path}
che costituiscono il tratto associato al checkpoint. Infine offre una serie di metodi da utilizzarsi per interagire con le risorse sottostanti.\\
I \textbf{Path} appena accennati vengono generati in fase di creazione del \textbf{Checkpoint} a partire dalle informazioni di base
reperite dal file di configurazione. Vengono generati tanti \textbf{Path} quanto la molteplicit\`{a} impostata del tratto. Ogni \textbf{Path} 
riceve valori di lunghezza che possono variare. Bisogna assicurare 1.5 m di larghezza per ogni macchina (approssimativamente). La prima
riceve i valori di base e viene virtualmente posizionata su un bordo del tratto. Le altre vengono posizionate man mano una di fianco all'altra,
quindi la lunghezza della traiettoria cresce in rapporto alla distanza dalla prima e all'angolo del tratto. Tutti i \textbf{Path} di un tratto
vengono incapsulati in una risorsa protetta denominata \textbf{Crossing} che ne regola l'accesso in mutua esclusione e offre i metodi di 
accesso pubblici.\\
Infine viene creata una corsia dei box coerente con la distanza fra i checkpoint ``prebox'' e l'``exitbox''. I tratti prima e dopo il checkpoint
dei box sono a molteplicit\`{a} uguale al numero di concorrenti. Questo perch\`{e} potenzialmente ai box potrebbero esserci contemporaneamente tutte
le auto. Inoltre si \`{e} deciso di posizionare il checkpoint del box in coincidenza con il traguardo. Quindi ogni box \`{e} anche un goal.\\\\
Verranno ora elencati metodi pi\`{u} rilevanti esposti dal \textbf{Checkpoint\_Synch} 
per poter mettere in pratica la strategia di attraversamento descritta nella sezione \ref{strategia_simulazione}:
\begin{description}
\item{\textbf{procedure Signal\_Arrival(CompetitorID\_In : INTEGER)}}\\
Il metodo marca nella coda del checkpoint l'arrivo effettivo del concorrente;
\item{\textbf{procedure Signal\_Leaving(CompetitorID\_In : INTEGER)}}\\
Il metodo marca nella coda del checkpoint l'uscita del concorrente;
\item{\textbf{procedure Set\_ArrivalTime(CompetitorID\_In : INTEGER; Time\_In : FLOAT)}}\\
Il metodo segna il tempo previsto di arrivo del concorrente nella coda del checkpoint;
\item{\textbf{procedure Remove\_Competitor(CompetitorID\_In : INTEGER)}}\\
Il metodo rimuove il competitor dalla coda. Ci\`{o} significa che l'id e il tempo del competitor non apparir\`{a} pi\`{u}
nella coda. Questo metodo \`{e} utilizzato, per esempio, quando un concorrente finisce la gara prima di altri
e deve quindi liberare i checkpoint per evitare di creare starvation;
\item{\textbf{function Get\_Time(CompetitorID\_In : INTEGER) return FLOAT}}
La funzione ritorna il tempo segnato sulla coda del concorrente con ID dato in input;
\item{\textbf{entry Wait\_Ready(Competitor\_ID : INTEGER)}}
Nel momento in cui un concorrente arriva fisicamente su un checkpoint, dopo aver marcato il suo arrivo
utilizza questo metodo per sapere quando arriva il suo turno per attraversare (viene cio\`{e} posizionato nella prima posizione della coda);
\item{\textbf{procedure Get\_Paths(Paths2Cross : out CROSSING\_POINT;  Go2Box : BOOLEAN)}}
Quando il concorrente sa di essere primo sulla coda del \textbf{Checkpoint}
(in seguito all'invocazione del metodo \underline{Wait\_Ready}), potr\`{a} invocare questa procedura per
ottenere l'insieme di \textbf{Path} che costituiscono il tratto e procedere alla valutazione. Il booleano ``Go2Box'', se valorizzato a ``true'',
impone al \textbf{Checkpoint\_Synch} di tornare (se presente) l'insieme di \textbf{Path} relativi alla corsia dei box.
Si \`{e} sicuri che nessuno star\`{a} effettuando operazioni sul tratto nel frattempo perch\`{e} tale azione da parte
degli altri concorrenti non \`{e} ammisibile fino a che il \textbf{Competitor} corrente non abbia segnalato la sua partenza dal checkpoint
tramite \underline{Signal\_Leaving}.
\end{description}
I metodi della risorsa \textbf{Crossing} invece servono a ritornare le caratteristiche di ogni \textbf{Path} (a partire dall'indice) e a aggiornare
l'istante temporale segnato nel path.\\\\
Infine sono offerti un insieme di metodi da utilizzare insieme al \textbf{RaceTrack\_Iterator} per navigare iterare il circuito. I metodi sono pi\`{u}
rilevanti sono:
\begin{description}
\item{\textbf{procedure Get\_CurrentCheckpoint(RaceIterator : in out RACETRACK\_ITERATOR; CurrentCheckpoint : out CHECKPOINT\_SYNCH\_POINT)}}\\
Per ottenere il checkpoint correntemente ``puntato'' dall'iteratore;
\item{\textbf{procedure Get\_NextCheckpoint(RaceIterator : in out RACETRACK\_ITERATOR; NextCheckpoint : out CHECKPOINT\_SYNCH\_POINT)}}\\
Per ottenere il checkpoint successivo. Nota Bene: il checkpoint dei box non \`{e} previsto essere ritornato da questo metodo. Per ottenere il
checkpoint dei box \`{e} necessario utilizzare \underline{Get\_BoxCheckpoint};
\item{\textbf{procedure Get\_PreviousCheckpoint(RaceIterator : in out RACETRACK\_ITERATOR; PreviousCheckpoint : out CHECKPOINT\_SYNCH\_POINT)}}\\
Per ottenere il checkpoint precedente, con le stesse regole del metodo precedente;
\item{\textbf{procedure Get\_ExitBoxCheckpoint(RaceIterator : in out RACETRACK\_ITERATOR; ExitBoxCheckpoint : out CHECKPOINT\_SYNCH\_POINT)}}\\
Per ottenere il checkpoint all'uscita della corsia dei box;
\item{\textbf{procedure Get\_BoxCheckpoint(RaceIterator : in out RACETRACK\_ITERATOR; BoxCheckpoint : out CHECKPOINT\_SYNCH\_POINT)}}\\
Per ottenere il checkpoint del box;
\item{\textbf{function Get\_Position(RaceIterator : RACETRACK\_ITERATOR) return INTEGER}}\\
Per tornare la posizione corrente dell'iteratore.
\end{description}
Non \`{e} stato necessario inserire \textbf{Racetrack} o il suo iteratore in una risorsa protetta perch\`{e} ogni concorrente disponde della
sua copia dell'\textbf{Racetrack\_Iterator}.
\newpage
\subsubsection{Stats}
\begin{center}
\begin{figure}[h!]
	\includegraphics[scale=0.50]{img/ClassDiagrams/StatsClassDiagram.jpg}
\caption{Class diagram - Stats}
\end{figure}
\end{center}
La componente \emph{Stats} \`{e} implicitamente suddivisa in 2 sottocomponenti: \textbf{OnboardComputer} e \textbf{CompetitionComputer}. 
Prima di discutere dei dettagli delle due, \`{e} necesario introdurre la struttura di una risorsa che viene utilizzata come
pacchetto per il trasporto degli aggiornamenti tra \emph{Competitor}, \textbf{OnboardComputer} e \textbf{CompetitionComputer}. La risorsa \`{e} 
un tipo record denominato \textbf{Competitor\_Stats}, contenente i seguenti dati:
\begin{itemize}
      \item\textbf{Time : FLOAT;}\\l'istante a cui fa riferimento l'aggiornamento
      \item\textbf{Checkpoint : INTEGER;}\\il checkpoint che introduceva il tratto che \`{e} stato attraversato (completato) all'istante \textsc{Time}
      \item\textbf{LastCheckInSect : BOOLEAN;}\\se true, il checkpoint \`{e} l'ultimo di un settore
      \item\textbf{FirstCheckInSect : BOOLEAN;}\\su treu, il checkpoint \`{e} il primo del settore
      \item\textbf{Sector : INTEGER;}\\il settore a cui appartiene il checkpoint
      \item\textbf{Lap : INTEGER;}\\la lap in cui a cui l'aggiornamento fa riferimento
      \item\textbf{GasLevel : FLOAT;}\\il livello di gas presente nel serbatoio al momento in cui il tratto \`{e} stato attraversato (completato)
      \item\textbf{TyreUsury : PERCENTAGE;}\\la percentuale di usura gomme al momento in cui il tratto \`{e} stato attraversato (completato)
      \item\textbf{BestLapNum : INTEGER;}\\la miglior lap fatta dal concorrente dall'inizio della gara all'istante \textsc{Time}
      \item\textbf{BestLaptime : FLOAT;}\\il tempo della miglior lap fatta dal concorrente dall'inizio della gara all'istante \textsc{Time}
      \item\textbf{BestSectorTimes : FLOAT\_ARRAY(1..3);}\\ogni indice dell'array indica un settore (quindi indice 1 indica il settore 1). Detto ci\`{o},
      l'array contiene il miglior tempo fatto dal concorrente per ogni settore dall'inizio della gara all'istante \textsc{Time}
      \item\textbf{MaxSpeed : FLOAT;}\\la massima velocit\`{a} raggiunta dall'inizio della gara all'istante \textsc{Time}
      \item\textbf{PathLength : FLOAT;}\\la lunghezza della traiettoria scelta per attraversare il tratto
\end{itemize}
Ora vediamo piu dettagliatamente le due sottocomponenti accennate precedentemente:
\begin{itemize}
\item \textbf{OnboardComputer}:\\
\`{e} un computer dedicato ad ogni singolo concorrente, denominato \textbf{OnboardComputer}. Ogni concorrente ne mantiene un'istanza che utilizza
per aggiornare le statistiche di checkpoint in checkpoint. Pi\`{u} precisamente, ogni concorrente possiede un'istanza di \textbf{Computer}, un record
che colleziona le statistiche del singolo concorrente e le informazioni statiche sulla configurazione della competizione.\\
Ogni \textbf{Computer} mantiene inoltre un riferimento ad una risorsa che nel diagramma \`{e} \textbf{Synch\_Info\_For\_Box}. Tale risorsa
viene utilizzata per mantenere la lista delle informazioni necessarie al box. \`{E} pi\`{u} che altro un meccanismo di ottimizzazione per avere
gli aggiornamenti sul singolo competitor immediatamente disponibili quando il \emph{Box} ne richiede.\\\\
Il \emph{Competitor} utilizza il metodo \underline{AddInfo} di \textbf{OnBoardComputer} 
per inviare le informazioni relative al tratto appena attraversato. Da quando tali informazioni sono sottomesse, vengono effettuati i seguenti
passaggi:
\begin{itemize}
\item se la fine del tratto coincide con la fine del settore, viene verificato il tempo impiegato per attraversare tale settore (facendo riferimento
anche ai dati passati) e se migliore di quello precedentemente salvato (in \textbf{Computer}), viene aggiornato quello vecchio.
\item se la fine del tratto coincide con la fine del settore vengono anche aggiornate le informazioni in \textbf{Synch\_Info\_For\_Box}. 
Si ricorda infatti che il \emph{Box} riceve le informazioni aggiornate alla fine di ogni settore.
\item se la fine del tratto corrisponde con la fine del giro, vengono aggiornato il miglior tempo di giro come fatto per i settori.
\item una volta effettuati tutti i controlli, le informazioni vengono impacchettate e inviate a \textbf{CompetitionComputer} per ulteriori
controlli e per essere salvate.
\end{itemize}
\item \textbf{CompetitionComputer}:\\
\`{e} invece il computer dedicato al calcolo delle statistiche globali, ovvero riguardanti tutti i partecipanti. Qualunque informazione 
che riguardi uno o pi\`{u} concorrenti \`{e} da richiedere a questa entit\`{a}. 
Il \textbf{CompetitionComputer} si appoggia a due risorse per l'archiviazione e l'organizzazione dei dati:
\begin{description}
\item{\textbf{All\_Competitor\_Stats\_Collection}}: la risorsa \`{e} destinata a mantenere la storia degli aggiornamenti di tutti i concorrenti. 
Ad ogni competitor \`{e} dedicato un array di \textbf{Synch\_Competitor\_Stats\_Handler}. Questa entit\`{a} serve ad racchiudere un \textbf{Competitor\_Stats}
nel corpo di una risorsa protetta. L'array \`{e} inizializzato con capacit\`{a} pari a N$^{\circ}$ Lap * N$^{\circ}$ Checkpoint, poich\`{e} le informazioni
vengono aggiunte ad ogni checkpoint. Ogni posizione dell'array quindi fa riferimento ad un checkpoint e l'array \`{e} intrinsecamente ordinato per 
tempo crescente.\\
\textbf{Synch\_Competitor\_Stats\_Handler} mette a disposizione un entry di get (\underline{Get\_All}) che si apre solo nel momento in cui
la risorsa \`{e} inizializzata. Questo permette di mettere in attesa i client che richiederanno informazioni non ancora disponibili.
\item{\textbf{SOCT\_Array}}: questo array \`{e} la parte pi\`{u} alta di una struttura utilizzata per il supporto alla creazione della classifica.
Nel punto pi\`{u} basso c'\`{e} \textbf{Classification\_Table}, unit\`{a} finalizzata a raccogliere i tempi di lap dei concorrenti. A gestire questa 
tabella virtuale c'\`{e} \textbf{Synch\_Ordered\_Classification\_Table}, che permette di mantenere la tabella ordinata in base ai tempi e offre un set
di metodi per il reperimento dei dati. Infine \textbf{SOCT\_Array} (SOCT sta per Synch Ordered Classification Table) \`{e} un array in cui
ogni posizione rimanda ad alla tabella della classifica della lap corrispondente all'indice. 
\end{description}
Tramite il metodo \underline{Add\_Stat}, il \textbf{Computer} di \textbf{OnboardComputer} invia i pacchetti con gli aggiornamenti. Prima
di salvare definitivamente un aggiornamento, viene verificato se fa riferimento ad un checkpoint di fine lap. In tal caso viene prima aggiornata
la tabella della classifica corrispondente alla lap appena percorsa con l'istante di tempo segnato. Successivamente l'aggiornamento
viene salvato nell'array di riferimento del concorrente, aprendo cos\`{i} la risorsa a chiunque la richieda o la stesse gi\`{a} richiedendo.\\\\
Quanto descritto costituisce le fondamenta di \textbf{CompetitionComputer}. Il metodi offerti navigano queste strutture per ottenere
i dati richiesti:
\begin{description}
\item{\textbf{procedure Get\_StatsByTime(Competitor\_ID : INTEGER; Time : FLOAT;  Stats\_In : out COMPETITOR\_STATS\_POINT);}}\\
fornisce il primo aggiornamento con tempo maggiore o uguale all'istante dato;
\item{\textbf{procedure Get\_StatsBySect(Competitor\_ID : INTEGER; Sector : INTEGER;  Lap : INTEGER;  Stats\_In : out COMPETITOR\_STATS\_POINT);}}\\
fornisce le statistiche del concorrente \textsc{Competitor\_ID} inerenti alla fine del settore richiesto nella lap richiesta;
\item{\textbf{procedure Get\_StatsByCheck(Competitor\_ID : INTEGER; Checkpoint : INTEGER; Lap : INTEGER; Stats\_In : out COMPETITOR\_STATS\_POINT);}}\\
fornisce le statistiche del concorrente \textsc{Competitor\_ID} inerenti al checkpoint e lap richiesti;
\item{\textbf{procedure Get\_BestLap(TimeInstant : FLOAT; LapTime : out FLOAT; LapNum : out INTEGER; Competitor\_ID : out INTEGER);}}\\
fornisce il miglior giro, il tempo di tale giro e il concorrente che fatto il record;
\item{\textbf{procedure Get\_BestSectorTimes(TimeInstant : FLOAT;  Times : out FLOAT\_ARRAY;  Competitor\_IDs : out INTEGER\_ARRAY; Laps : out INTEGER\_ARRAY);}}\\
fornisce il miglior tempo per ogni settore con i concorrenti che hanno fatto il record.
\item{\textbf{procedure Get\_LapClassific(Lap : INTEGER; TimeInstant : FLOAT; CompetitorID\_InClassific : out INTEGER\_ARRAY\_POINT; 
Times\_InClassific : out FLOAT\_ARRAY\_POINT; LappedCompetitors\_ID : out INTEGER\_ARRAY\_POINT; LappedCompetitor\_CurrentLap : out INTEGER\_ARRAY\_POINT);}}\\
dato l'istante di tempo in input, il metodo fornisce la classifica pi\`{u} aggiornata con i tempi per quell'istante, compresi i concorrenti doppiati in ordine
di posizione e la lap che stanno percorrendo. I concorrenti che invece non sono doppiati ma devono ancora finire la lap a cui la classifica si
riferisce non vengono inclusi nella lista.
\end{description}
\end{itemize}
Il \textbf{Competition\_Monitor}, (come vedremo poi pi\`{u} in dettaglio \`{e} una sottocomponente di \emph{Monitor}) fa riferimento al \textbf{CompetitionComputer} per 
ottenere le informazioni di competizione.
\newpage
\subsubsection{Monitor}
\begin{center}
\begin{figure}[h!]
	\includegraphics[scale=0.50]{img/ClassDiagrams/MonitorClassDiagram.jpg}
\caption{Class diagram - Monitor}
\end{figure}
\end{center}
\emph{Monitor} si suddivide in due sottocomponenti: \textbf{Box\_Monitor} e \textbf{Competition\_Monitor}. La prima è dedicata alle informazioni
esposte dal \emph{Box}, l'altra a quelle esposte dalla \emph{Competition}.\\
\textbf{Box\_Monitor} produce i dati interrogando la componente \emph{Box}, precisamente \textbf{Box\_Data}, unità che vedremo in seguito,
tramite il metodo \underline{Get\_Info}, grazie al quale si ottiene un oggetto che contiene le informazioni inerenti all'ultimo settore
completato dal concorrente (comprese le medie di consumo) e, se disponibile la strategia calcolata dal \emph{Box} per la lap successiva.\\\\
\texbtf{Competition\_Monitor} si appoggia a \emph{Stats} per reperire le informazioni richieste. I metodi che espone sono:
\begin{description}
\item\textbf{procedure Get\_CompetitionInfo( TimeInstant : FLOAT; ClassificationTimes : out Common.FLOAT\_ARRAY\_POINT; XMLInfo : out Unbounded\_String.Unbounded\_String);}}\\
la procedura inizializza la stringa con le informazioni di competizione in formato XML. Tali informazioni riguardano il posizionamento
dei concorrenti all'istante dato (quale tratto stanno percorrendo e in che lap), le migliori performance fino all'istante dato e la classifica
aggiornata all'ultima lap in corso, con tempi e concorrenti doppiati. Il metodo si appoggia ai metodi forniti da \textbf{Competition\_Stats}
per reperire le inormazioni necessarie.
\item\textbf{procedure Get\_CompetitorInfo(lap : INTEGER; sector : INTEGER ; id : INTEGER; time : out FLOAT; updString : out Unbounded\_String.Unbounded\_String);}}\\
la procedura fornisce le informazioni di un concorrente aggiornate al settore e lap richieste. Il metodo si appoggia a \textbf{OnBoardComputer}
per reperire le informazioni date, precisamente utilizzando il metodo \underline{Get_BoxInfo} utilizzando come parametro un riferimento
al \textbf{Computer} del concorrente di cui si richiedono le informazioni.
\end{description}
Queste classi non comunicano direttamente con i loro clienti, ma comunicano tramite un intermediario: \textbf{Box\_Monitor\_Radio} e 
\textbf{Competition\_Monitor\_Radio}, ai quali viene demandato il compito di gestire le comunicazione distribuito tramite Polyorb (usando
il protocollo CORBA).
\newpage
\subsubsection{Box}
\begin{center}
\begin{figure}[h!]
	\includegraphics[scale=0.50]{img/ClassDiagrams/BoxClassDiagram.jpg}
\caption{Class diagram - Box}
\end{figure}
\end{center}
\newpage
\subsubsection{Configurator}
\begin{center}
\begin{figure}[h!]
	\includegraphics[scale=0.50]{img/ClassDiagrams/ConfiguratorClassDiagram.jpg}
\caption{Class diagram - Configurator}
\end{figure}
\end{center}
\newpage
\subsubsection{Screen}
\begin{center}
\begin{figure}[h!]
	\includegraphics[scale=0.50]{img/ClassDiagrams/ScreenClassDiagram.jpg}
\caption{Class diagram - Screen}
\end{figure}
\end{center}
%Diagrammi delle classi per ogni componente
%

% Elenco dei task con descrizione
\subsection{Risorse attive}
% Elenco risorse condivise con descrizione
\subsection{Risorse passive}
\begin{itemize}
\item{Risorse protette}
\item{Altre risorse}
\end{itemize}
%"Analisi della concorrenza"
\subsection{Analisi della concorrenza}
	%. analisi dell'interazione risorse e task (senza menzionare la distribuzione)
	%. dimostrazione assenza di racecondition
	%. dimostrazione assenza di starvation
\begin{itemize}
\item{Interazione tra risorse condivise e task}
\item{Assenza di racecondition}
\item{Assenza di starvation}
\end{itemize}
%"Distribuzione"
\subsection{Distribuzione}
	%. Elenco risorse distribuite
	%. Interazione risorse distribuite
	%. Misure di fault tolerance
\begin{itemize}
\item{Componenti distribuite}%Con motivazione
\item{Interazione fra le componenti distribuite}
\item{Misure di fault tolerance}
\end{itemize}
% Inizializzazione gara
\subsection{Inizializzazione competizione}
% Stop gara
\subsection{Stop competizione}
