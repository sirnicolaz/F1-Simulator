'%%%Architettura alto livello%%%
\subsection{Architettura ad alto livello}
Nella seguente sezione verr\`{a} illustrata l'architettura ad alto livello del sistema sviluppato, 
escludendo i dettagli implementativi e legati al linguaggio.
% Componenti del sistema
\subsubsection{Componenti di sistema}
Le principali componenti del sistema sono:
\begin{description}
\item{Competition}
La \emph{Competition} \`{e} l'unit\`{a} atta ad orchestrare l'avvio e la conclusione della corsa. Tale componente, dunque, \`{e} stata concepita per 
offrire le seguenti funzionalit\`{a}:
\begin{itemize}
\item Configurazione parametri di gara:
	\begin{itemize}
		\item numero di giri;
		\item numero di concorrenti;
		\item circuito;
	\end{itemize}
\item Gestione della sessione di iscrizione e accettazione concorrenti (configurati a livello della componente \emph{Box})
\item Avvio delle componenti necessarie al monitoraggio della gara (quali ad esempio \emph{Monitor})
\item Avvio controllato della competizione vera e propria nel momento in cui tutti i prerequisiti di inizio sono soddisfatti, ovvero:
\begin{itemize}
\item la competizione \`{e} stata configurata correttamente;
\item le componenti di controllo e gestione della competizione sono attive e in attesa di comandi;
\item i concorrenti sono stati correttamente registrati alla competizione e in attesa di partire;
\end{itemize}
\end{itemize}
\item{Competitor}
Il \emph{Competitor} \`{e} l'entit\`{a} pensata ad svolgere la gara. \`{E} caratterizzato dalle seguenti sotto-componenti:
\begin{itemize}
\item \textbf{Auto}, ovvero tutte le caratteristiche fisiche legate all'auto, ovvero:
	\begin{itemize}
		\item motore;
		\item capacit\`{a} del serbatoio;
		\item massima accelerazione;
		\item massima velocit\`{a};
		\item gomme montate (mescola, modello, tipo);
		\item livello usura gomme;
		\item livello della benzina nel serbatoio;
	\end{itemize}
\item \textbf{Guidatore}, cio\`{e} le informazioni che descrivono pi\`{u} dettagliatamente il concorrente in gara:
	\begin{itemize}
		\item nome e cognome pilota;
		\item nome scuderia
	\end{itemize}
\item \textbf{Strategia}, ovvero la strategia che sta adottando il pilota, suggerita dai box e dinamica nel corso della gara:
	\begin{itemize}
		\item style di guida, variabile tra conservativo, normale, aggressivo, a seconda dello stato della macchina e delle
			previsioni fatte dai box
		\item numero di lap prima del pit stop
		\item addizionalmente, quando viene fatto un pit stop, la strategia determina anche quali siano le nuove gomme da montare,
		 	la quantit\`{a} di benzina da avere nel serbatoio e il tempo impiegato per il pit stop.
	\end{itemize}
\end{itemize}
Tutte queste informazioni insieme creano quello che viene definito il concorrente di gara. 
Tali informazioni verranno poi usate nel corso della gara per:
	\begin{itemize}
		\item scegliere al momento giusto la miglior traiettoria da seguire, in base alla presenza o meno di altri concorrenti
			nelle vicinanze e alla difficolt\`{a} del tratto;
		\item fornire costantemente aggiornamenti sul suo stato (tramite una parte del modulo \emph{Stats}, informando il computer di bordo
			riguardo a:
			\begin{itemize}
				\item livello di usura gomme;
				\item livello di benzina;
				\item checkpoint attraversato con tempo di arrivo;
				\item insieme al checkpoint verranno aggiornate le informazioni relative a settore e lap;
				\item velocit\`{a} massima raggiunta.
			\end{itemize}
		\item contattare ad ogni giro il box per ottenere una strategia aggiornata;
		\item se suggerito dai box, effettuare un pitstop;
		\item ritirarsi dalla gara una volta che le condizione dell'auto non permettano di poter correre ulteriormente;
		\item banalmente, continuare a correre fino alla fine delle lap prestabilite, dopodich\`{e} fermarsi.
	\end{itemize}
\item{Circuit}
\item{Stats}
\item{Box}
\item{Radio}
\item{Monitor}
\end{description}
 - lista delle componenti con descrizione ad alto livello del loro scopo
 - se necessario fornire un diagramma delle componenti
% Interazione fra le componenti
\subsubsection{Interazione fra le componenti}
 - descrivere ad alto livello l'interazione fra componenti e se necessario aiutarsi con OCR cards
% Strategia adottata per la correttezza temporale
\subsubsection{Strategia adottata per la correttezza temporale}
% Dimostrazione dell'assenza di stallo
\subsubsection{Assenza di stallo}
%%% Architettura in dettaglio %%
\subsection{Architettura in dettaglio}
% Elenco dei task con descrizione
\subsubsection{Risorse attive}
% Elenco risorse condivise con descrizione
\subsubsection{Risorse passive}
\begin{itemize}
\item{Risorse protette}
\item{Altre risorse}
\end{itemize}
%"Analisi della concorrenza"
\subsubsection{Analisi della concorrenza}
	%. analisi dell'interazione risorse e task (senza menzionare la distribuzione)
	%. dimostrazione assenza di racecondition
	%. dimostrazione assenza di starvation
\begin{itemize}
\item{Interazione tra risorse condivise e task}
\item{Assenza di racecondition}
\item{Assenza di starvation}
\end{itemize}
%"Distribuzione"
\subsubsection{Distribuzione}
	%. Elenco risorse distribuite
	%. Interazione risorse distribuite
	%. Misure di fault tolerance
\begin{itemize}
\item{Componenti distribuite}%Con motivazione
\item{Interazione fra le componenti distribuite}
\item{Misure di fault tolerance}
\end{itemize}
% Inizializzazione gara
\subsection{Inizializzazione competizione}
% Stop gara
\subsection{Stop competizione}
