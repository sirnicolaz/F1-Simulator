\section{Glossario}
\begin{list}{}
\item \textbf{A}\\
\textsc{Antenna}: a receiver put on the toll lane -check-in -check-out that read the the information of a toll tag to recognize a customer and notify the lane controller of his arrival.\\
\item \textbf{B}\\
\textsc{Bank}: a generic bank that represents the entity intended to handle the toll lane customers' account when the toll lane system perform a transaction.\\
\textsc{Barrier}: a physical obstacle positioned on each check-in and check-out. It is intended to prevent an unauthorized customer to pass through.\\
\item \textbf{C}\\
\textsc{Concorrente}: l'entità costituita da pilota e auto che partecipa alla gara correndo sul circuito.\\
\textsc{Cashier (review)}: employee who works in every toll lane station check-in whose task can be summurized in the lsit below:
\begin{itemize}
	\item receive a customer who wants to pass through the toll lane;
	\item calculate the price based on the vehicle type;
	\item take right amount the money and eventually register the customer;
\end{itemize}
\textsc{Card reader}: the device used to read the customer's credit card and perform the transaction with a bank.\\
\textsc{Check-in}: the entering point of the toll lane. There are two kind of Check-in:
\begin{itemize}
\item Express, where a customer che pass through using a toll tag;
\item Normal, where a customer has to stop and pay (using cash or credit card) before being able to pass.
\end{itemize}
Each Check-in has a barrier.\\
\textsc{Check-out}: the exit point of the toll lane. there are two kind of Check-out:
\begin{itemize}
\item Express, where a customer can pass through using a toll tag;
\item Normal, where a customer has to stop and has his ticket validated by the ticket reader before being abel to pass.
\end{itemize}
Each Check-out has a barrier.\\
\textsc{Customer}: the client of the toll lane.\\
\item \textbf{D}
\item \textbf{E}
\textsc{Enterprise server}:\\
\item \textbf{F}
\item \textbf{G}
\item \textbf{H}
\item \textbf{I}
\item \textbf{J}
\item \textbf{K}
\item \textbf{L}\\
\textsc{Lane Controller}: a software system conceived to offer a communication bridge between the input interfaces of the system (like the touchscreen, the ticket reader ...) and the other components (like the station server, the printer ...).\\
\item \textbf{M}
\item \textbf{N}
\item \textbf{O}
\item \textbf{P}\\
\textsc{Printer}: a physical device used by check-in employee to print tickets.\\
\item \textbf{Q}
\item \textbf{R}
\item \textbf{S}\\
\textsc{Station Server}: a server offering the following services:
\begin{itemize}
\item register the toll lane system customer's activities (through the communication with lane controller);
\item store the information about the current prices for the toll lane system service;
\item provide the enterprise server with all the information concerning the toll lane station activities.
\end{itemize}
\item \textbf{T}\\
\textsc{Ticket}: it's sold to the customer at the toll lane station or in a specialised center. It gives the customer the possibility to use the toll lane. It's used at the check-out to leave the toll lane.\\
\textsc{Ticket reader}: physical device used to recognize a valid ticket and eventually notify the lane controller.\\
\textsc{Toll tag}: a device with an RFID chip used by the customer to pass through the express check-in (and check-out) once readed by the antenna.\\
\textsc{Touchscreen}: the physical interface used by the cashier to execute his activities.\\
\item \textbf{U}
\item \textbf{V}
\item \textbf{X}
\item \textbf{Y}
\item \textbf{W}
\item \textbf{Z}
\end{list}
