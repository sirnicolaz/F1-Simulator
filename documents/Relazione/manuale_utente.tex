\section{Manuale utente}
\subsection{Installazione}
\subsection{Avvio}
\subsection{Terminazione}
\subsection{Interfaccia box}
Nella prima schermata che appare vanno inseriti i dati relativi al concorrente che si intende registrare.
\begin{enumerate}
\item Impostare il nome del concorrente
\item Impostare il cognome del concorrente
\item Impostare la scuderia del concorrente
\item Impostare il livello di seriet\`{a} del box (come descritto da \ref{box})
\item Impostare la massima capacit\`{a} del serbatoio
\item Impostare il livello di benzina iniziale
\item Impostare il tipo di gomme montate
\item Impostare la mescola delle gomme montate (le gomme a mescola morbida si consumano pi\`{u} rapidamente)
\item Impostare lo stile di guida del concorrente 
\item Impostare la massima accelerazione della macchina
\item Impostare la massima velocit\`{a} raggiungibile
\item Inserire il corbaloc che fa riferimento al Registration Hanlder
\item Pulsante per avviare la competizione
\item Pulsante per riportare la gui allo stato iniziale
\end{enumerate}
Una volta inseriti i dati e premuto il pulsante di avvio della competizione comparir\`{a} una finestra con al suo interno 
\begin{enumerate}
\item Pannello dei consumi medi del concorrente con dati relativi alla benzina (in litri al kilometro) e dell'usura delle gomme (in percentuale relativo a 1 km)
\item Pannello con il log della gara aggiornato a ogni fine settore con dati di tempo di gara, livello di benzina e usura delle gomme
\item Pannello con altre informazioni statiche sulla gara (configurazione iniziale, stile di guida)
\end{enumerate}
\subsection{Interfaccia competizione}
La prima interfaccia relativa alla competizione serve per impostare la competizione stessa. I parametri da settare sono:
\begin{enumerate}
\item File xml con il tracciato
\item Numero di concorrenti richiesti
\item Numero di giri previsti
\end{enumerate}
Una volta settati i parameteri e schiacciato il pulsante \emph{Start Competition} verr\`{a} avviato una tv che presenta le informazioni sulle iscrizioni nella parte bassa della finestra. Ogni volta che un concorrente si iscrive compare nella gui e una volta raggiunto il numero di concorrenti stabiliti la gara si avvia e la gui comincia ad offrire i dati disponibili, come spiegato nel paragrafo \ref{interfacciaTv}
\subsection{Interfaccia TV}
\label{interfacciaTv}
L'interfaccia della tv consiste in un pannello iniziale con al suo interno un cronometro che scandisce il tempo di aggiornamento delle informazioni seguito da un pannello con le informazioni sul miglior giro e sui migliori tempi nei settori.
L'intervallo per il tempo che scorre \`{e} impostabile se si tratta di una tv configurata tramite il \emph{configuratorScreen}, prestabilito (e molto basso) se si tratta della tv avviata dalla competizione.
Il pannello centrale offre la visualizzazione di due tabelle. La tabella a destra si riferisce all'ultima classifica disponibile mentre a sinistra viene visualizzata la classifica del giro precedente (escluso al primo giro di gara dove viene presentata una tabella vuota).
Nella parte bassa della finestra viene presentato un log della competizione rappresentando a ogni istante di tempo posizione nella pista, numero di checkpoint, numero di settore e giro per ogni concorrente.
La componente di avvio dell'interfaccia TV pu\`{o} essere l'interfaccia di competizione oppure una schermata di configurazione dove va inserito il corbaloc del Monitor della competizione e impostato il tempo di refresh per il reperimento delle informazioni.