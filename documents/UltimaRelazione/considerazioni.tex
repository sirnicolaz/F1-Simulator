\section{Considerazioni varie sulla nostra soluzione}
\subsection{Determinismo vs Non determinismo}
La scelta del non determismo ovvero
      la soluzione proposta durante il colloquio avrebbe garantito coerenza temporale per il sistema concorrente locale ( su assunzione di istruzioni macchina eseguite, da noi non fatte) ma avrebbe inevitabilmente prodotto un non determinismo dato successivamente dai ritardi di rete che non prendono in considerazione l’orologio di sistema della competizione. La soluzione suggerita in fase di colloquio prevede che ogni task (concorrente) che gareggia, abbia la corsa scandita dall’orologio assoluto. Ogni istante di risveglio corrisponde all’istante “reale” in cui tale concorrente avrebbe intrapreso l’attraversamento del nuovo segmento. Per quanto riguarda i box (distribuiti), la nostra soluzione prevede che il concorrente instauri una comunicazione sincrona via radio col box per ottenere informazioni sul come procedere con la corsa. Comunicazione che fallirebbe solo in caso di mancanza di rete. E’ lecito pensare che con la soluzione proposta durante il colloquio, invece, ciò non sia possibile: i ritardi dati da una RPC sono nell’ordine dei ms e potrebbero aumentare con condizioni di rete pessime. Il task che stia quindi calcolando l’istante di risveglio per l’esecuzione successiva verrebbe influenzato da questo ritardo. Nella realtà sarebbe come se il pilota si fermasse mentre comunica con il box per poi ripartire, il che non è plausibile. Questo può essere risolto cambiando la direzione di comunicazione: il box comunica in modo asincrono eventuali modifiche di strategia al concorrente. Il thread concorrente, in prossimità dei box o in qualunque altro momento, dovrebbe alterare la sua strategia in caso ci sia stata una comunicazione da parte dei box, e questo occuperebbe anche un numero limitato di istruzioni. Questo comunque introdurrebbe un livello di non determinismo che non si voleva avere inizialmente. I ritardi di rete non vengono considerati dal sistema su cui viene eseguita la gara, essendo tale sistema basato su sospensione assoluta. Un ritardo di rete si potrebbe quindi ripercuotere in un mancato arrivo della nuova strategia al momento richiesto. E questo cambierebbe un esecuzione da  quella precedente. Abbiamo pertanto scelto di progettare il sistema in modo che i casi come quello sopra citato venissero ridotti, fino a presentarsi solo in caso la comunicazione di rete venisse tagliata (in questo caso si è SCELTO di far procedere la competizione in una direzione diversa, ma si sarebbe potuto comunque far attendere il concorrente fino alla riconnessione per garantire un esito di competizione uguale a quelli avvenuti sotto le stesse condizioni).
\subsection{Esperienza utente}
Per quanto rigurarda l’esperienza utente, non è da tralasciare il fatto che si è puntato di più sulla correttezza delle informazioni presentate, fornendo la possibilità di velocizzare/rallentare a piacimento l’esecuzione. Il tempo che si vede scorrere negli schermi è comunque solidale con quello relativo alla competizione anche se non perfettamente mappabile con quello reale che scorre. Questo è principalmente dovuto ai ritardi di rete, che anche con la soluzione proposta, avrebbero fatto visualizzare il tempo in maniera non omogenea. Infatti le informazioni visualizzate se arrivassero in ritardo (dovuto alla rete) sembrerebbe che il tempo che scorre della competizione non sia solidale con quello reale.